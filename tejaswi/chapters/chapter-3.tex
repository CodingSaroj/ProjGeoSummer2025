\chapter{Affine Geometry}

\section{Affine Space}

A set $\varepsilon$ is endowed with the structure of an affine space by a vector space $E$ and a
mapping $\Theta$ that associates a vector of $E$ with any ordered pair of points in
$\varepsilon$,

\begin{eqnarray*}
  \varepsilon \times \varepsilon &\longrightarrow & E \\
  (A,B) &\longmapsto & \overrightarrow{AB}
\end{eqnarray*}

such that:

\begin{itemize}
  \item[-]{for any point $A$ of $\varepsilon$, the partial map $\Theta_A : B \mapsto \overrightarrow{AB}$
      is a bijection from $\varepsilon$ to $E$.}
  \item[-]{for any points $A$, $B$, and $C$ in $\varepsilon$, we have $\overrightarrow{AB}=
      \overrightarrow{AC}+\overrightarrow{CB}$.}
\end{itemize}

The vector space $E$ is the direction of $\varepsilon$, or its underlying vector space. The
elements of $\varepsilon$ are called points, and the dimension of the vector space $E$ is called
the dimension of $\varepsilon$. \cite{audin}\\

\section{Fundamental Theorem of Affine Geometry}

\section{Affine Congruence of Conics}
