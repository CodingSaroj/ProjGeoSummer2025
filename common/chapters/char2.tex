\chapter{Conics in Characteristic 2 Fields} \label{ch:char2}

Given a conic $Ax^2 + Bxy + Cy^2 + Dx + Ey + F = 0$, we've classified it by
writing it in matrix form as
\[
    \begin{bmatrix}x & y\end{bmatrix}
    \begin{bmatrix}A & \frac{1}{2}B \\ \frac{1}{2}B & C\end{bmatrix}
    \begin{bmatrix}x \\ y\end{bmatrix}
    +
    \begin{bmatrix}D & E\end{bmatrix}
    \begin{bmatrix}x \\ y\end{bmatrix}
    + F = 0
\]
and diagonalizing the symmetric matrix to obtain an orthonormal basis within
which, the conic only has no $xy$ term. However, this method no longer works if
we're working in $\F^2$ such that $\ch(\F)=2$ as $1+1=2=0$ and hence, we
can't divide by 2.

\vspace{1ex}

In this chapter, we'll classify the conics in finite fields with characteristic 2
and investigate the conic groups that arise from them. We'll state the following
theorem which will be used heavily throughout the chapter:

\begin{theorem} \label{thm:field_sq}
    If $\F$ is a finite field with $\ch(\F)=2$, then
    $\forall\,a\in\F\ \exists\,b\in\F$ such that $b^2=a$. We'll write
    $b=\sqrt{a}$.
\end{theorem}

\begin{proof}
    Suppose $|\F|=q$. Since $\ch(\F)=2$, $q$ is even. So, $|\F^\times|=q-1$ is
    odd. As $\F^\times$ is cyclic for any finite field $\F$
    \cite[Prop.~9.18]{dummit}, we have $\F^\times=\langle g \rangle$ for some
    $g\in\F^\times$.
    \vspace{1ex}

    Take any $a\in\F^\times$. Then, $a=g^k$ for some $k\in\Z/(q-1)\Z$. Now, $2$
    has a multiplicative inverse in $\Z/(q-1)\Z$ since $q-1$ is odd. Hence,
    $\exists\,b\in\F^\times$ such that $b=g^{2^{-1}k}$. It is easy to see that
    $b^2=a$.
\end{proof}

\section{Classification}

Consider a non-degenerate, non-singular conic in a finite characteristic 2 field
$\F$ given by
\[ Ax^2 + Bxy + Cy^2 + Dx + Ey + F = 0 \]
where not all of $A,B$ and $C$ are zero.

\subsection*{Case 1: $A \neq 0, B \neq 0$ and $C \neq 0$}

Apply the affine transformation
$(x,y)\mapsto\left(\frac{x}{\sqrt{A}},\frac{y}{\sqrt{C}}\right)$ and take
$H=\frac{B}{\sqrt{A}\sqrt{C}}$ to get the following form
\[ x^2 + Hxy + y^2 + \frac{D}{\sqrt{A}}x + \frac{E}{\sqrt{C}}y + F = 0 \]
Further applying the affine transformation 
$(x,y)\mapsto\left(x+\frac{E}{H\sqrt{C}},y+\frac{D}{H\sqrt{A}}\right)$ gives
\[ x^2 + Hxy + y^2 + K = 0 \]
where
$K = F + \frac{D^2}{H^2 A} + \frac{E^2}{H^2 C} + \frac{DE}{H\sqrt{A}\sqrt{C}}$.
Now, applying the affine transformation
$(x,y)\mapsto\left(\frac{\sqrt{K}}{H}x,\frac{\sqrt{K}}{H}y\right)$ gives
\[ \frac{K}{H^2}x^2 + \frac{K}{H^2}xy + \frac{K}{H^2}y^2 + K = 0\]
We can multiply by $\frac{H^2}{K}$ since $K=0$ will make the conic singular. Thus,
taking $L=H^2$ we have

\begin{center}
    \fbox{$x^2 + xy + y^2 + L = 0$}
\end{center}

\noindent
where $L$ can be any non-zero element of $\F$ due to Theorem \ref{thm:field_sq}.

\subsection*{Case 2: $A = 0, B \neq 0$ and $C \neq 0$}

Apply the affine transformation $(x,y)\mapsto(x,x+y)$ to get
\[ (B+C)x^2 + Bxy + Cy^2 + (D+E)x + Ey + F = 0 \]
If $B \neq C$, we can proceed as Case 1. Otherwise, we have
\[ Bxy + By^2 + Dx + Ey + F = 0\]
Applying the affine transformation
$(x,y)\mapsto\left(\frac{x}{\sqrt{B}},\frac{y}{\sqrt{B}}\right)$ and take
$H=\frac{B}{\sqrt{C}}$ to get the following form
\[ xy + y^2 + \frac{D}{\sqrt{B}}x + \frac{E}{\sqrt{B}}y + F = 0 \]
Further applying the affine transformation 
$(x,y)\mapsto\left(x+\frac{E}{\sqrt{B}},y+\frac{D}{\sqrt{B}}\right)$ gives
\[ xy + y^2 + K = 0 \]
where $K = F + \frac{D^2}{B} + \frac{DE}{B}$. Now, applying the affine
transformation $(x,y)\mapsto(x+y,y)$ gives
\[ xy + K = 0 \]
We can now proceed as Case 3.

\subsection*{Case 3: $A = 0, B \neq 0$ and $C = 0$}

Applying the affine transformation
$(x,y)\mapsto\left(x+\frac{E}{B},y+\frac{D}{B}\right)$ gives
\[ Bxy + H = 0 \]
where $H = F + \frac{DE}{B^2}$. Dividing by $B$ and taking $K=\frac{H}{B}$, we get
\[ xy + K = 0\]
Finally, applying the affine transformation $(x,y)\mapsto(\sqrt{K}x,\sqrt{K}y)$
and dividing by $K$ ($K=0$ implies conic is singular) results in

\begin{center}
    \fbox{$xy + 1 = 0$}
\end{center}

\subsection*{Case 4: $A \neq 0, B = 0$ and $C \neq 0$}

Apply the affine transformation
$(x,y)\mapsto\left(\frac{x}{\sqrt{A}},\frac{y}{\sqrt{C}}\right)$ to get the
following form
\[ x^2 + y^2 + \frac{D}{\sqrt{A}}x + \frac{E}{\sqrt{C}}y + F = 0 \]
Further applying the affine transformation $(x,y)\mapsto(x,x+y)$ gives
\[ y^2 + Hx + Ky + F = 0 \]
where $H=\left(\frac{D}{\sqrt{A}} + \frac{E}{\sqrt{C}}\right)$ and
$K=\frac{E}{\sqrt{C}}$. We can proceed as Case 5 from here.

\subsection*{Case 5: $A = 0, B = 0$ and $C \neq 0$}

Applying the affine transformation
$(x,y)\mapsto\left(x,\frac{y}{\sqrt{C}}\right)$ to get the
following form
\[ y^2 + Dx + \frac{E}{\sqrt{C}}y + F = 0 \]
Note that since $D=0$ gives a quadratic equation in $y$, this case corresponds to
a degenerate conic. Hence, we can assume $D \neq 0$. So, applying the affine
transformation
$(x,y)\mapsto\left(\frac{x}{D} + \frac{Ey}{D\sqrt{C}} + \frac{F}{D},y\right)$ gives

\begin{center}
    \fbox{$y^2 + x = 0$}
\end{center}

\noindent
Thus, upto affine congruence there are 3 classes of non-degenerate, non-singular
conics in finite fields of
characteristic two:
\begin{enumerate}[label=\Roman*.]
    \item $y^2 + x = 0$
    \item $xy + 1 = 0$
    \item $x^2 + xy + y^2 + L = 0\quad\forall\,L\in\F^\times$
\end{enumerate}
From here on, we'll refer to these as Type I, Type II and Type III conics
respectively.
\vspace{1ex}

\noindent
Note that Type I and Type II have equations similar to parabola and hyperbola.
Further, since $x^2 + 1 = 0$ has a solution in any field of characteristic two,
Theorem \ref{thm:hyp_ell_corr} gives us that ellipses and hyperbolae will be
affine congruent. Hence, all the non-degenerate conics we're used to in $\R$ are
contained in the Type I and Type II cases. Type III, however, is a new class that
appears only in the case of characteristic two fields.


\section{Conic Groups}

For Type I and Type II conics, we can achieve a similar parametrization as done
for any field with characteristic not two in Chapter \ref{ch:conics}. This gives
us the groups corresponding to Type I and Type II conics to be isomorphic to
$\langle\F,+\rangle$ and $\langle\F^\times,\cdot\rangle$.
\vspace{1ex}

\noindent
For Type III conics, we have to consider a quadratic field extension $\F(\alpha)$
as a two dimensional vector space over $\F$ with an ordered basis $\{1,\alpha\}$.
Note that such an $\alpha$ is guaranteed to exist as the finite field of order
$|\F|^2$ has a subfield isomorphic to $\F$ since $|\F|\mid|\F|^2$
\cite[\S14.3]{dummit}.
Suppose $\alpha$ is the root of the equation $x^2 + bx + c$ and hence
$\alpha^2 = b\alpha+c$.

Note that for some fixed $a=a_1+a_2\alpha\in\F(\alpha)$, multiplying any
$x=x_1+x_2\alpha\in\F(\alpha)$ by $a$ is an $\F$-linear map since
\begin{align*}
    ax &= (a_1+a_2\alpha)(x_1+x_2\alpha) \\
       &= a_1 x_1 + (a_1 x_2 + a_2 x_1)\alpha + a_2 x_2 \alpha^2 \\
       &= (a_1 + a_2\alpha)x_1 + (a_2 c + a_1\alpha + a_2 b\alpha)x_2 \\
       &= \begin{bmatrix}a_1 & a_2 c \\ a_2 & a_1 + a_2 b\end{bmatrix}
          \begin{bmatrix}x_1 \\ x_2\end{bmatrix}
   \end{align*}
We'll consider a map $N\colon\F(\alpha)\to\F$ which sends $z=x+y\alpha\in\F(\alpha)$ to
the determinant of the above matrix for multiplying by $a$ i.e.

\[
    N(z) = N(x + y\alpha)
    = \left|\begin{matrix}x & cy \\ y & x + by\end{matrix}\right|
    = x(x + by) + c y^2
    = x^2 + cy^2 + bxy
\]

This map is known as the field norm on $\F(\alpha)$. It is easy to see that the
equation $N(z)=k$ for some $k\in\F^\times$ corresponds to a conic in $\F^2$ that
is Type III (since $A=1$, $B=b$ and $C=c^2$). Further, $N$ can be thought of as a
group homomorphism from $\F(\alpha)^\times$ to $\F^\times$ since $N(1)=1$ and
\begin{align*}
    N((x+y\alpha)(z+w\alpha)) &= N(xz+yw\alpha^2+(xw+yz)\alpha) \\
                              &= N(xz+cyw+(xw+yz+byw)\alpha) \\
                              &= x^2 (z^2 + cw^2 + bzw) + cy^2 (z^2 + cw^2 + bzw) \\
                              &\quad+ bxy (z^2 + cw^2 + bzw) \\
                              &= (x^2 + cy^2 + bxy) (z^2 + cw^2 + b zw) \\
                              &= N(x+y\alpha)N(z+w\alpha)
\end{align*}
For any $z\in\F(\alpha)$ that satisfies $N(z)=k$, every element of the coset
$(\ker N)z$ also satisfies $N(z)=k$. In fact, these are the only solutions as
cosets are either equal or disjoint. Hence, the order of the conic group of Type
III is $|\ker N|$.

\vspace{1ex}

This is where the investigation ends. There are two main directions to complete
this theory -- first is classifying the structure of the group of Type III conics
-- and second is looking at the case of infinite fields with characteristic two.
