\chapter{Affine Geometry}

\section{Affine space}

\begin{definition}
    Given a vector space $\vec{X}$ over $\mathbb{F}$, its set of points $X$ and
    an operation $+: X\times\vec{X} \to X$ such that $\forall\,\vec{v},\vec{w}\in\vec{X}$
    and $\forall\,p\in\vec{X}$,
    \begin{enumerate}
        \item $p + \vec{0} = p$
        \item $p + (\vec{v} + \vec{w}) = (p + \vec{v}) + \vec{w}$
        \item $\theta_{p} : \vec{X} \to X$ given by
            $\theta_{p}(\vec{v}) = p + \vec{v}$ is a bijection.
    \end{enumerate}
    Then $X$ is called an affine space with underlying vector space $\vec{X}$.
\end{definition}

%\begin{notation}
%    For a triple $(A,\vec{A},+)$ denoted as $\AffF{n}{F}$, we'll use
%    $a\in\Aff{n}$ to mean $a \in A$ and $\vec{a}\in\AffF{n}{F}$ to mean
%    $\vec{a}\in\vec{A}$.
%\end{notation}

\noindent
Due to the third point above, we have the following definition:

\begin{definition}
    Given an affine space $X$, for any $\,a,b \in X$,
    \[ b - a := \theta_a^{-1}(b) \]
\end{definition}

\section{Affine frames and coordinates}

\begin{definition}
    An $(n+1)$-tuple $(p_0,\vec{v}_1,\dots,\vec{v}_n)$ where $p_0 \in X$ and
    $\{\vec{v}_1,\dots,\vec{v}_n\}$ is a basis of $\vec{X}$ is called an affine
    frame.
\end{definition}

\noindent
Given $p \in X$ and an affine frame $(p_0,\vec{v}_1,\dots,\vec{v}_n)$ of $X$, if
$p-p_0=c_1\vec{v}_1+\dots+c_n\vec{v}_n$, then $p$ is said to have coordinates
$(c_1,\dots,c_n)$ in that frame.

\section{Affine transformation}

\begin{definition}
    Given an affine space $X$, a funtion $f: X \to X$ is said to be an affine
    transformation if $\exists\,\vec{f}\in\End(\vec{X}): \vec{f}(b-a)=f(b)-f(a)\ \forall\,a,b \in X$.
\end{definition}

\begin{notation}
    We denote the set of affine transformations over $X$ as $\A(X)$ and the set of
    invertible affine transformations over $X$ as $\GA(X)$.
\end{notation}

\begin{theorem} \label{thm:aff_lin_rep}
    Given $f\in\A(X)$, $\vec{f}$ is unique. Further, given some $p_0 \in X$,
    $\exists!\,b\in{A}$ such that $f(p)=b+\vec{f}(p-p_0)\ \forall\,p \in X$.
\end{theorem}

\begin{proof}
    Suppose $\vec{f}_1,\vec{f}_2\in\End(\vec{X})$ such that for any $a,b \in X$
    \begin{align*}
        \vec{f}_1(b-a)=f(b)-f(a) \\
        \vec{f}_2(b-a)=f(b)-f(a)
    \end{align*}
    Assume $\exists\,\vec{v}\in\vec{X}: \vec{f}_1(\vec{v})\neq\vec{f}_2(\vec{v})$.
    For some $a \in X$, we have $\theta_a(\vec{v}) \in X$ such that
    $\theta_a(\vec{v})-a=\theta_a^{-1}(\theta_a(\vec{v}))=\vec{v}$. This means
    \[
        \vec{f}_1(\vec{v}) = \vec{f}_1(\theta_a(\vec{v})-a)
        = f(\theta_a(\vec{v}))-f(a)
        = \vec{f}_2(\theta_a(\vec{v})-a)= \vec{f}_2(\vec{v})
    \]
    This is a contradiction. Hence, our assumption that such a $\vec{v}$ exists
    must be wrong and so, $\vec{f}_1=\vec{f}_2$.
    \vspace{1ex}

    \noindent
    Fixing some $p_0 \in X$, we have
    $\vec{f}(p-p_0)=f(p)-f(p_0)\ \forall\,p \in X$. So,
    \[ f(p)=f(p_0)+\vec{f}(p-p_0)\ \forall\,p \in X \]
    Hence, $b=f(p_0)$. For some $b_1,b_2 \in X$ and $b_1 \neq b_2$, assume
    \[ f(p)=b_1+\vec{f}(p-p_0)\ \forall\,p \in X \]
    \[ f(p)=b_2+\vec{f}(p-p_0)\ \forall\,p \in X \]
    Note that
    \[ b_1=b_1+(\vec{f}(p-p_0)-\vec{f}(p-p_0))=(b_1+\vec{f}(p-p_0))-\vec{f}(p-p_0)=f(p)-\vec{f}(p-p_0) \]
    \[ b_2=b_2+(\vec{f}(p-p_0)-\vec{f}(p-p_0))=(b_2+\vec{f}(p-p_0))-\vec{f}(p-p_0)=f(p)-\vec{f}(p-p_0) \]
    Hence, $b_1=b_2$.
\end{proof}

\section{Properties of Affine Transformations}

\begin{definition}
    Given $a,b \in X$, we define the line passing through $a$ and $b$ as
    \[ \ell_{ab}:=\{a+t(b-a): t\in\mathbb{F}\} \]
\end{definition}

\begin{definition}
    Two lines $\ell_{ab}$ and $\ell_{pq}$ are said to be parallel if
    $b-a=k(p-q)$ for some $k\in\mathbb{F}$. We write this
    as $\ell_{ab}\parallel\ell_{pq}$.
\end{definition}

\begin{theorem}
    Consider $f\in\GA(X)$ and $\ell_{ab}$ for some $a,b \in X$. Then,
    \[ \exists\,p,q \in X: f(\ell_{ab})=\ell_{pq} \]
\end{theorem}

\begin{proof}
    Fixing $p_0=a$ in Theorem \ref{thm:aff_lin_rep}, we have $p \in X$ such that
    \[ f(a+t(b-a))=p+\vec{f}(t(b-a))=p+t\vec{f}(b-a)\ \forall\,t\in\mathbb{F} \]
    Since $\vec{v} \mapsto p+\vec{v}$ is a bijection, we have $q \in X$ such that
    $q-p=\vec{f}(b-a)$. Thus
    \[ f(a+t(b-a))=p+t(q-p)\ \forall\,t\in\mathbb{F} \]
    i.e. $f(\ell_{ab})=\ell_{pq}$.
\end{proof}

The above theorem can be interpreted as the following statement:
\vspace{1ex}

\begin{center}
    \fbox{\emph{Affine transformations take straight lines to straight lines.}}
\end{center}
\vspace{1ex}

\begin{theorem}
    For any $f\in\GA(X)$,
    \[ \ell_{ab} \parallel \ell_{pq} \implies f(\ell_{ab}) \parallel f(\ell_{pq}) \]
\end{theorem}

\begin{proof}
    Since $\ell_{ab}\parallel\ell_{pq}$, we have $b-a=k(q-p)$ for some
    $k\in\mathbb{F}$. Using Theorem \ref{thm:aff_lin_rep}, we can write
    \begin{align*}
        f(\ell_{ab}) &= \{f(a + t(b-a)): t\in\mathbb{F}\} \\
                     &= \{c + \vec{f}(a + t(b-a) - p_0): t\in\mathbb{F}\} \\
                     &= \{c + \vec{f}((a-p_0)+t(b-a)): t\in\mathbb{F}\} \\
                     &= \{c + \vec{f}(a-p_0) + t\vec{f}(b-a): t\in\mathbb{F}\}
    \end{align*}
    Similarily,
    $f(\ell_{pq}) = \{c + \vec{f}(p - p_0) +t\vec{f}(q-p): t\in\mathbb{F}\}$.
    Now,
    \begin{align*}
        b - a = k(q - p) \implies \vec{f}(b-a) = k\vec{f}(q - p)
    \end{align*}
    By definition, this means that $f(\ell_{ab}) \parallel f(\ell_{pq})$.

\end{proof}

The above theorem can be interpreted as the following statement:
\vspace{1ex}

\begin{center}
    \fbox{\emph{Affine transformations take parallel lines to parallel lines.}}
\end{center}
\vspace{1ex}

\noindent
If the underlying vector space $\vec{X}$ of an affine space $X$ has a norm 
$\norm{\cdot}$ defined on it, we have the following theorem:

\begin{theorem}
    Given $f\in\GA(X)$, a line $\ell_{ac}$ and any $b\in\ell_{ac}$ such that
    $b \neq a$ and $b \neq c$, we have
    \[ \frac{\norm{b-a}}{\norm{c-b}}=\frac{\norm{f(b)-f(a)}}{\norm{f(c)-f(b)}} \]
\end{theorem}

\begin{proof}
    Since $b\in\ell_{ac}$, let $b=a+t_0(c-a)$. Now,
    \begin{align*}
        \frac{\norm{b-a}}{\norm{c-b}} = \frac{|t_0| \norm{c-a}}{|1-t_0|\norm{c-a}}
                                      = \left|\frac{t_0}{1-t_0}\right|
    \end{align*}
    Using Theorem \ref{thm:aff_lin_rep} with $p_0=a$, we have
    $f(x)=p+\vec{f}(x-a)$ for some $p \in X$. So,
    \begin{align*}
        f(a) &= p+\vec{f}(a-a) = p \\
        f(b) &= p+\vec{f}(a+t_0 (c-a)-a) = p + t_0 \vec{f}(c-a) \\
        f(c) &= p+\vec{f}(c-a) \\
    \end{align*}
    Hence,
    \begin{align*}
        \frac{\norm{f(b)-f(a)}}{\norm{f(c)-f(b)}}
        = \frac{|t_0| \norm{\vec{f}(c-a)}}{|1-t_0|\norm{\vec{f}(c-a)}}
        = \left|\frac{t_0}{1-t_0}\right|
    \end{align*}
\end{proof}

The above theorem can be interpreted as the following statement:
\vspace{1ex}

\begin{center}
    \fbox{\emph{Affine transformations preserve the ratio of distances of 3
    collinear points.}}
\end{center}
\vspace{1ex}

\section{Fundamental theorem of Affine Geometry}

\begin{theorem}
    If $A_0,A_1,\dots,A_n,B_0,B_1,\dots,B_n \in X$ such that
    $\{A_1-A_0,\dots,A_n-A_0\}$ and $\{B_1-B_0,\dots,B_n-B_0\}$ are linearly
    independent where $n = \dim \vec{X}$, then
    \[ \exists!\,f\in\GA(X): f(A_i) = B_i\ \forall\,i\in\{0,1,\dots,n\} \]
\end{theorem}

\begin{proof}
    Let $\vec{v}_i=A_i-A_0$ and $\vec{w}_i=B_i-A_0\ \forall\,i\in\{1,2,\dots,n\}$.
    Clearly, both $\beta_1=\{\vec{v}_1,\dots,\vec{v}_n\}$ and
    $\beta_2=\{\vec{w}_1,\dots,\vec{w}_n\}$ form a basis for $\vec{X}$. In fact,
    there are unique linear transformations $\vec{f}_1,\vec{f}_2\in\GL(\vec{X})$
    such that $\vec{f}_1(\vec{v}_i)=\vec{e}_i$ and
    $\vec{f}_2(\vec{w}_i)=\vec{e}_i\ \forall\,i\in\{1,2,\dots,n\}$ where
    $\{\vec{e}_1,\dots,\vec{e}_n\}$ is the standard basis of $\vec{X}$.
    \vspace{1ex}

    \noindent
    Consider the affine transformations $f_1,f_2\in\GA(X)$ given by
    \begin{align*}
        f_1(p) = O + \vec{f}_1(p-A_0)\ \forall\,p \in X \\
        f_2(p) = O + \vec{f}_1(p-B_0)\ \forall\,p \in X 
    \end{align*}
    Now, $f_1(A_0)=f_2(B_0)=O$ and
    $f_1(A_i)=f_2(B_i)=O+\vec{e}_i\ \forall\,i\in\{1,2,\dots,n\}$. Since $f_1$ and
    $f_2$ are invertible, it is easy to see that $f=f_2^{-1}f_1$ satisfies
    $f(A_i)=B_i\ \forall\,i\in\{0,1,\dots,n\}$.
    \vspace{1ex}

    \noindent
    Next, we need to prove that $f$ is unique. Suppose there are two affine
    transformations $g_1,g_2\in\GA(X)$ that satisfy
    $g_1(A_i)=g_2(A_i)=B_i\ \forall\,i\in\{0,1,\dots,n\}$ but $\exists\,q_0 \in X$
    such that $g_1(q_0) \neq g_2(q_0)$.
    \vspace{1ex}

    \noindent
    From Theorem \ref{thm:aff_lin_rep}, picking $p_0=A_0$,
    $\exists!\,b_1,b_2 \in X$ such that $\forall\,q \in X$
    \begin{align*}
        g_1(p) = b_1 + \vec{g}_1(p-A_0) \\
        g_2(p) = b_2 + \vec{g}_2(p-A_0)
    \end{align*}
    Since $g_1(A_0)=g_2(A_0)=B_0$, we have $b_1=b_2$. Further using
    \[ g_1(A_i)=g_2(A_i)=B_i\ \forall\,i\in\{1,2,\dots,n\} \]
    we get the relations
    \[ \vec{g}_1(\vec{v}_i)=\vec{g}_2(\vec{v}_i)\ \forall\,i\in\{1,2,\dots,n\} \]
    But note that $\beta_1$ is a basis of $\vec{X}$. Thus for any
    $\vec{a}\in\vec{X}$, we have scalars $c_1,\dots,c_n$ such that
    $\vec{a}=c_1\vec{v}_1+\dots+c_n\vec{v}_n$. Hence,
    \begin{align*}
        \vec{g}_1(\vec{a}) = c_1\vec{g}_1(\vec{v}_1)+\dots+c_n\vec{g}_1(\vec{v}_2)
        = c_1\vec{g}_2(\vec{v}_1)+\dots+c_n\vec{g}_2(\vec{v}_2)
        = \vec{g}_2(\vec{a})\ \forall\,\vec{a}\in\vec{X}
    \end{align*}
    So, $b_1=b_2$ and $\vec{g}_1=\vec{g}_2$. But this contradicts that
    $\exists\,q_0 \in X: g_1(q_0) \neq g_2(q_0)$. Hence, $g_1=g_2$.
\end{proof}

Intuitively, this theorem says that there exists an affine transformation in
$\GA(X)$ which takes an $n$-simplex in an affine space $X$ with $\dim \vec{X}=n$
to another $n$-simplex in $X$. Note that an $n$-simplex is a generalization of the
concept of triangles and tetrahedra in 2D and 3D respectively. In particular, a
triangle is a 2-simplex and a tetrahedron is a 3-simplex. So, if we consider the
affine space $\mathbb{R}^2$, this theorem says that there is an affine
transformation that takes any triangle to any other triangle. We can also state it
as
\vspace{1ex}

\begin{center}
    \fbox{\emph{All triangles in $\mathbb{R}^2$ are affine-congruent.}}
\end{center}
\vspace{1ex}

\noindent
In general, we say two figures are affine-congruent if there is an invertible
affine transformation taking one to the other.
