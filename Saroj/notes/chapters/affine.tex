\chapter{Affine Geometry}

\section{Affine space}

\begin{definition}
    Given a vector space $\vec{X}$ over $\mathbb{F}$, its set of points $X$ and
    an operation $+: X\times\vec{X} \to X$ such that $\forall\,\vec{v},\vec{w}\in\vec{X}$
    and $\forall\,p\in\vec{X}$,
    \begin{enumerate}
        \item $p + \vec{0} = p$
        \item $p + (\vec{v} + \vec{w}) = (p + \vec{v}) + \vec{w}$
        \item $\theta_{p} : \vec{X} \to X$ given by
            $\theta_{p}(\vec{v}) = p + \vec{v}$ is a bijection.
    \end{enumerate}
    Then $X$ is called an affine space with underlying vector space $\vec{X}$.
\end{definition}

%\begin{notation}
%    For a triple $(A,\vec{A},+)$ denoted as $\AffF{n}{F}$, we'll use
%    $a\in\Aff{n}$ to mean $a \in A$ and $\vec{a}\in\AffF{n}{F}$ to mean
%    $\vec{a}\in\vec{A}$.
%\end{notation}

\noindent
Due to the third point above, we have the following definition:

\begin{definition}
    Given an affine space $X$, for any $\,a,b \in X$,
    \[ b - a := \theta_a^{-1}(b) \]
\end{definition}

\section{Affine frames and coordinates}

\begin{definition}
    An $(n+1)$-tuple $(p_0,\vec{v}_1,...,\vec{v}_n)$ where $p_0 \in X$ and
    $\{\vec{v}_1,...,\vec{v}_n\}$ is a basis of $\vec{X}$ is called an affine
    frame.
\end{definition}

\noindent
Given $p \in X$ and an affine frame $(p_0,\vec{v}_1,...,\vec{v}_n)$ of $X$, if
$p-p_0=c_1\vec{v}_1+...+c_n\vec{v}_n$, then $p$ is said to have coordinates
$(c_1,...,c_n)$ in that frame.

\section{Affine transformation}

\begin{definition}
    Given an affine space $X$, a funtion $f: X \to X$ is said to be an affine
    transformation if $\exists\,\vec{f}\in\End(\vec{X}): \vec{f}(b-a)=f(b)-f(a)\ \forall\,a,b \in X$.
\end{definition}

\begin{notation}
    We denote the set of affine transformations over $X$ as $\A(X)$ and the set of
    invertible affine transformations over $X$ as $\GA(X)$.
\end{notation}

\begin{theorem}
    Given $f\in\A(X)$, $\vec{f}$ is unique. Further, given some $p_0 \in X$,
    $\exists!\,b\in{A}$ such that $f(p)=b+\vec{f}(p-p_0)\ \forall\,p \in X$.
\end{theorem}

\begin{proof}
    Suppose $\vec{f}_1,\vec{f}_2\in\End(\vec{X})$ such that for any $a,b \in X$
    \begin{align*}
        \vec{f}_1(b-a)=f(b)-f(a) \\
        \vec{f}_2(b-a)=f(b)-f(a)
    \end{align*}
    Assume $\exists\,\vec{v}\in\vec{X}: \vec{f}_1(\vec{v})\neq\vec{f}_2(\vec{v})$.
    For some $a \in X$, we have $\theta_a(\vec{v}) \in X$ such that
    $\theta_a(\vec{v})-a=\theta_a^{-1}(\theta_a(\vec{v}))=\vec{v}$. This means
    \[
        \vec{f}_1(\vec{v}) = \vec{f}_1(\theta_a(\vec{v})-a)
        = f(\theta_a(\vec{v}))-f(a)
        = \vec{f}_2(\theta_a(\vec{v})-a)= \vec{f}_2(\vec{v})
    \]
    This is a contradiction. Hence, our assumption that such a $\vec{v}$ exists
    must be wrong and so, $\vec{f}_1=\vec{f}_2$.
    \vspace{1ex}

    \noindent
    Fixing some $p_0 \in X$, we have
    $\vec{f}(p-p_0)=f(p)-f(p_0)\ \forall\,p \in X$. So,
    \[ f(p)=f(p_0)+\vec{f}(p-p_0)\ \forall\,p \in X \]
    Hence, $b=f(p_0)$. For some $b_1,b_2 \in X$ and $b_1 \neq b_2$, assume
    \[ f(p)=b_1+\vec{f}(p-p_0)\ \forall\,p \in X \]
    \[ f(p)=b_2+\vec{f}(p-p_0)\ \forall\,p \in X \]
    Note that
    \[ b_1=b_1+(\vec{f}(p-p_0)-\vec{f}(p-p_0))=(b_1+\vec{f}(p-p_0))-\vec{f}(p-p_0)=f(p)-\vec{f}(p-p_0) \]
    \[ b_2=b_2+(\vec{f}(p-p_0)-\vec{f}(p-p_0))=(b_2+\vec{f}(p-p_0))-\vec{f}(p-p_0)=f(p)-\vec{f}(p-p_0) \]
    Hence, $b_1=b_2$.
\end{proof}

\section{Fundamental theorem of Affine Geometry}
